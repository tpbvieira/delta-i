\documentclass[10pt]{article}

\newcommand{\Fix}[1]{\textbf{[[#1]]}}
\newcommand{\Comment}[1]{}
\newcommand{\CodeIn}[1]{\begin{small}\texttt{#1}\end{small}}
\newcommand{\Email}[1]{{\normalsize\textit{#1}}}
\newcommand{\newurl}[1]{\begin{scriptsize}\texttt{#1}\end{scriptsize}}
\newcommand{\etal}{\emph{et al.\/}}
\newcommand{\dE}{\Comment{$\Delta$}Delta Execution}
\newcommand{\DE}{DE}
\newcommand{\de}{DE}

%%
\newcommand{\lt}{$\langle{}$}
\newcommand{\gt}{$\rangle{}$}
\newcommand{\mybull}[1]{\noindent$\bullet$\hspace{0.5ex}\textbf{#1}}
 % macros file

\title{Applying Delta Execution on Mutation Testing}	

\author{Marcelo d'Amorim, David Schuler} % used by \maketitle
\date{November 9 , 2009}                 % used by \maketitle
\begin{document}
\maketitle                               % automatic title!

Two important issues in mutation testing are the time needed to
execute all mutations and presence of equivalent
mutations.  \DE{} has the potential to address both of these issues.
It addresses the first by executing multiple mutations in parallel
whenever possible. The second issue is addressed by comparing state of 
mutated and non mutated executions, as the difference between these states
allows quantify the impact of mutants on state, which can be
used to prioritize non-equivalent mutations.

\section{Background}

\subsection{Delta Execution}

\DE{}~(\DEA{})~\cite{damorim-tse-2008} provides a way to execute
programs with sets of states: the interpretation of one operation in
delta mode potentially reads from (writes to) several individual
states.  Consider, for illustration, a binary search tree
implementation with an \CodeIn{add(int)} operation.  The execution of
the following code fragment in delta mode builds 4 trees
\emph{simultaneously}.   The braces denote delta objects, i.e., objects that
merge several different individual values.

\begin{quote}
\begin{verbatim}
{bst}.add({1,2,3,4}); {bst}.add({2,3,4,5})
\end{verbatim}
\end{quote}

Assume that the \CodeIn{bst} object is initially empty.  The execution
of this fragment will follow the same execution path and add a child,
storing values \CodeIn{\{2,3,4,5\}}, to the right of the root node.
Unfortunately, sometimes it is not possible to follow the same
execution path.  Consider, for example, the delta object
\CodeIn{\{0,1,4,5\}} as input to the second call to \CodeIn{add}.  In
that case, \DEA{} needs to \emph{split} execution: one path results in
the addition of values \CodeIn{\{0,1\}} to the left of the root node,
another in the addition of \CodeIn{\{4,5\}} to the right.  After the
call, \DEA{} needs to \emph{merge} the state sets together.

The initial goal of the technique was to speed up explicit-state model
checking.  \DEA{} can take advantage of the control flow similarity
across different states to optimize time.  \DEA{} can also take
advantage of data similarity for efficient data access: in the case
above, all four individual states share the \CodeIn{bst} object and
only one integer suffices to encode the size of all trees.  Very
important to note is that one execution in delta mode is expensive.
Split and merge, in particular, are costly operations.  Speedup
depends on a number of factors.  Specially, the number of states and
potential for sharing.

\subsection{Mutation Testing}
Mutation testing assesses the quality of a test suite by inserting
artificial bugs into a program, and checking whether the test suite
finds them.  Mutation testing has two issues.  First, the \emph{time
  needed to check all mutations} --- the test suite has to be run for
every mutation. Second, the inability to automatically identify
\emph{equivalent mutations}, which are mutations that change the
syntax of the program but do not change its semantics.


\section{Combining Delta Execution and Mutation Testing}

A mutant schemata is a representation of the program that includes all mutants.
It uses a special\Comment{int? A Property => String} variable to identify the
enabled mutant. We propose to run each test on the mutant schemata with \DE{}.
Conceptually, that corresponds to running a test on all mutants simultaneously.

For illustration, suppose the original program uses the expression
\CodeIn{x > y} and that the mutant schemata includes a mutant for that
expression that replaces \CodeIn{>} with \CodeIn{<=}.  Additionally,
consider that the mutant schemata contains 5 mutants and the mutant
above corresponds to mutant number 3.  \DE{} of one test on the mutant
schemata will have 6 individual states: 1 corresponding to the
original program and 5 for the mutants.  The evaluation of the
expression above on the mutant schemata with values \CodeIn{[1]} for
\CodeIn{x} and \CodeIn{[2,2,2,2,2,3]} for \CodeIn{y} results in
\CodeIn{[f,f,f,t,f,f]}.  Few notes: (i) we use brackets to highlight
the importance of indexing state ids, (ii) the value 0 indexes the
original program state, (iii) the value of \CodeIn{y} was already
infested by mutation 5 at the expression evaluation (see value 3), and
(iv) a mutation does not necessarily require split.

\subsection{Goals}

\noindent\textbf{Speed up Mutation Testing.}  
Delta Execution can potentially execute several mutations in parallel.
In principle, speedup depends on the number of mutations per test and
potential for sharing data (and consequently control flow).
Informally, a one-state execution in delta mode is slower than a
standard execution and a multi-state execution in delta mode is very
likely slower if no sharing is observed.  A potential speedup to
explore is to ignore states associated to mutants with no observed
impact.  \Comment{please check if this still makes
  sense. -M. PART1:Therefore, the state will be split for every
  mutation, as the mutated and unmutated code is executed on the
  current state.  A potential speed up are states that can be merged
  again, e.g the mutation did not affect the program
  state. Furthermore the execution that happens before a mutation does
  not have to be repeated for every mutation.  PART2:How does delta
  execution deal with state derivations? E.g. a mutated execution
  takes different a control flow path than the normal execution.  PART
  3:How would the split of states at mutations be implemented? --M
  paragraph still makes sense --DS}\\

\noindent\textbf{Measure Impact.}
The impact of a mutation on state is helpful to improve a test
suite~\cite{schuler-issta-2009}.\Comment{please, let me know if there
  is other goal. -M} To that end, one can compare original and
individual state differences across one delta state.  We conjecture
that the higher the difference between non-mutated and mutated states
the higher the chance to observe the change with a test
execution.\Comment{don't understand this sentence.  thought that state
  change implies semantic changes. do you mean observable change?
  change in control flow? -M Yes, I meant an observable
  change. E.g. The program can have different state at some point but
  produces the same result, or only non-accessible values are changed
  that do not affect the result.-DS} Furthermore, the impact can
provide hints on how to improve the test suite. For example, one can
use information of state difference reachable from a test to augment
that test with state assertions.  \Comment{this is really nice! -M}\\

\noindent\textbf{Detect equivalent executions.} \DEA{} expresses one state for
each mutant plus one state for the original program. It is therefore feasible to
compare states, after the activation of a mutation and at the end of a run. If
there is no difference between the states, this suggests that either the
mutation is equivalent or the input data was not chosen carefully enough.
\Comment{careful: I guess the mutation can be
  activated several times.}\Comment{i guess the former approach also
  has the same limitations.  the quality of daikon invariant is
  limited by the  quality of operational profiles.}\\

\Comment{i guess we need to make sure the mutation is also
  unreachable. -M}

\Comment{I think it will not be feasible to prove that a mutation is
  equivalent since we only know that there is no state change for the
  input data provided by the test. There could exist other inputs that
  cause a state change.  However it might be a good indicator of
  equivalence. -D}

\Comment{i guess my comment was misplaced and confusing.  it was not
  about finding a mutant equivalent in the general case.  but in the
  case of a specific test.  that could help to optimize execution, not
  to rule out the mutant. -M}

\noindent\textbf{Partition Mutations.}
\sloppy Similarity of impact is a criteria to partition mutants:
mutations that end up in similar states, have similar effects in the
program.  One can conjecture that one test might detect all similar
mutations or none. This information is valuable to guide a programmer
in the improvement of a test suite.  He only
needs to focus on one representative of each partition.\\
    
\section{Concrete Steps}
\begin{itemize}
\item Apply delta execution on mutation testing.  \DEA{} has two
  implementations~\cite{damorim-tse-2008}: one that operates on JPF
  state and another in Java state.  The initial Java implementation
  used an Eclipse plugin to instrument code: user needs to select each
  compilation unit for instrumentation.  Justification of using
  Eclipse was re-use of standard refactorings.  Recent updates: batch
  of instrumentation visitors (using independent parser),
  simplification of infra-structure (some operations used in
  state-space exploration are unnecessary in mutation), cleanup and
  documentation.  In addition, current framework needs to be extended
  to support neglected operations: (i) native code, and (ii)
  exceptions raised with data access (e.g., arithmetic exception and
  null pointer exception).  Justification for these missing pieces:
  TSE paper focused on speedup.

  \Comment{should we comment about current status of implementation
    here?  -M Yes, I would suggest so --DS}
\item Measure speed up improvements
\item Compare different states using the delta state.
\end{itemize}

\bibliographystyle{abbrv}
\bibliography{testing}

\end{document}
% End of document.



% LocalWords:  bst ints multi david unmutated DS daikon JPF plugin refactorings
% LocalWords:  dereference TSE
